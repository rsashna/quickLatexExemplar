\documentclass[12pt,twoside,a4paper]{article}
%more classes:
% article for articles in scientific journals, presentations, short reports, pro-gram documentation, invitations, ...
% book has neat features, look for more at https://www.overleaf.com/learn/latex/headers_and_footers
% slides uses spaces out rounded sansserrif letters
% Beamer slides but sized down too
%Also columns with onecolumn,twocolumn
% landscape to change orientation of pg

% \usepackage{fontspec} %font change
% Times New Roman
% \setromanfont[
% BoldFont=timesbd.ttf,
% ItalicFont=timesi.ttf,
% BoldItalicFont=timesbi.ttf,
% ]{times.ttf}
% Arial
% \setsansfont[
% BoldFont=arialbd.ttf,
% ItalicFont=ariali.ttf,
% BoldItalicFont=arialbi.ttf
% ]{arial.ttf}
% \setromanfont{Times New Roman}
%   \setsansfont{Arial}
%   \setmonofont[Color={0019D4}]{Courier New}
\usepackage[dvipsnames]{xcolor}
% own colours can be picked with rgb/cymk like so \definecolor{nameOfClr}{RGB}{0, 117, 92}
% can take tints/shades of colours like so
\colorlet{TurquoiseLighter2}{Turquoise!40!}

\usepackage{color}
\usepackage{xcolor}

%language selector: \usepackage[english]{babel}
\usepackage{multicol}
\setlength{\columnsep}{1cm}

\usepackage{ragged2e}%for alignment purposes

\setlength{\columnseprule}{1pt} %creates a line b/w cols
\def\columnseprulecolor{\color{gray}}

\setlength{\parindent}{1em} %paragraph indentation size and space b/w paragraphs
\setlength{\parskip}{1em}

% make "C++" look pretty when used in text by touching up the plus signs
\newcommand{\spacerCmd}
{C\nolinebreak[4]\hspace{-.05em}\raisebox{.22ex}{\footnotesize\bf ++}}

\usepackage{tabularx} %for making less-simple tables


\usepackage{hyperref}
\hypersetup{
    colorlinks=true,
    linkcolor=blue,
    filecolor=magenta,
    urlcolor=cyan,
}
%.............................................%

\begin{document}

\begin{center}
  \section*{A Quick(er) Intro to \LaTeX}
\end{center}

\section*{This is a Section}

\subsection*{This is a Subsection about the \textbackslash}

\# \$ \% \^{} \& \_ \{ \} \~word \~{} \textbackslash
\newline \newline Since \textbackslash s are used in commands, the 'textbackslash' keyword is used instead
\newline Also use \% to comment out %TODO add something here

\subsection*{Italics and font styles}
Look at this \textsl{slanted} word
\newline or this \textit{italicised} word %also works with \emph{} which toggles the style
\newline Look at this \textbf{bold} word \newline And \underline{underlined} word
\newline Aaand \textbf{\textit{\underline{all of the above}}}
\newline \newline Add some \textcolor{blue}{blue coloured} text, or even some
\textcolor{Turquoise}{fancy turquoise coloured} text, with some
\textcolor{TurquoiseLighter2}{ligher hues}
\newpage

\subsection*{Changing number of Columns in a Section}
\begin{multicols}{2}
  [
This just needs to be in a single column. This just needs to be in a single column. This just needs to be in a single column. This just needs to be in a single column. This just needs to be in a single column. This just needs to be in a single column.
]
\noindent This needs to be in double columns. This needs to be in double columns.
\newline This needs to be in double columns. This needs to be in double columns.
\newline This needs to be in double columns. This needs to be in double columns.
\newline This needs to be in double columns. This needs to be in double columns.
\newline This needs to be in double columns. This needs to be in double columns.
\newline This needs to be in double columns. This needs to be in double columns.

\end{multicols}

\subsection*{Another Column Ex}
\par This just needs to be in a single column. This just needs to be in a single column. This just needs to be in a single column. This just needs to be in a single column. \par This just needs to be in a single column. This just needs to be in a single column.

\begin{multicols}{3}
  \setlength{\columnsep}{0.5cm}
  \setlength{\columnseprule}{0pt} %remove prev sep

  This needs to be in triple columns. This needs to be in triple columns. This needs to be in triple columns. This needs to be in triple columns. This needs to be in triple columns. This needs to be in triple columns.
  \newline This needs to be in triple columns. This needs to be in triple columns.
  \newline This needs to be in triple columns. This needs to be in triple columns.
  \newline This needs to be in triple columns. This needs to be in triple columns.
  \newline This needs to be in triple columns. This needs to be in triple columns.
  \newline This needs to be in triple columns. This needs to be in triple columns.
  \newline This needs to be in triple columns. This needs to be in triple columns.
  \newline This needs to be in triple columns. This needs to be in triple columns.
\end{multicols}

\newpage

\begin{center}
\subsection*{Text Alignment for Paragraphs}
This is a centered paragraph. This is a centered paragraph. This is a centered paragraph. This is a centered paragraph. This is a centered paragraph. This is a centered paragraph. This is a centered paragraph. This is a centered paragraph. This is a centered paragraph. This is a centered paragraph.
\end{center}

\begin{flushright}
This is a right aligned paragraph. This is a right aligned paragraph. This is a right aligned paragraph. This is a right aligned paragraph. This is a right aligned paragraph. This is a right aligned paragraph. This is a right aligned paragraph. This is a right aligned paragraph. This is a right aligned paragraph. This is a right aligned paragraph.
\end{flushright}

\begin{flushleft}
This is a left aligned paragraph. This is a left aligned paragraph. This is a left aligned paragraph. This is a left aligned paragraph. This is a left aligned paragraph. This is a left aligned paragraph. This is a left aligned paragraph. This is a left aligned paragraph. This is a left aligned paragraph. This is a left aligned paragraph.
\end{flushleft}

%other alighnment options \raggedleft \raggedright
\centering
Alternate command to align text. Alternate command to align text. Alternate command to align text. Alternate command to align text. Alternate command to align text. \justify Alternate command to align text. Alternate command to align text. Alternate command to align text. Alternate command to align text. Alternate command to align text. Alternate command to align text.

\newpage
\subsection*{Lines and Tables}

\vspace{40pt}

\hrule height 0.1pt \relax
Making some ruled lines
\hrule height 0.5pt \relax
Making some ruled lines
\hrule height 1pt \relax
Making some ruled lines
\hrule height 3pt \relax
Making some ruled lines
\hrule height 10pt \relax

\subsection*{Simple Table Structures}

\begin{tabular}{ c |c| c }
 someData & someData & someData \\
 \hline
 someData & someData & someData \\
 someData & someData & someData
\end{tabular}

\begin{tabular}{ |c|c|c| }
\hline
 someData & someData & someData \\
 someData & someData & someData \\
 someData & someData & someData \\
\hline
\end{tabular}

\subsection*{A Less-Simple Table Structure}
\begin{tabularx}{0.8\textwidth} {
  | >{\raggedright\arraybackslash}X
  | >{\centering\arraybackslash}X
  | >{\raggedleft\arraybackslash}X | }
 \hline
 moreData & moreData & moreData \\
 \hline
 moreData & moreData & moreData \\
\hline
\end{tabularx}
\newline since there are too many table formats to list, c/p from \href{https://www.overleaf.com/learn/latex/tables}{here instead}
\end{document}
